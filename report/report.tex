\documentclass{article}
\usepackage{amsmath} % Нужно для элементов математики.
\usepackage[utf8]{inputenc}
\pagestyle{empty}

\usepackage[usenames,dvipsnames,svgnames,table]{xcolor}
\usepackage{tikz-timing}[2009/05/15]
\usepackage{multicol}
\usepackage[T2A]{fontenc}
\usepackage[russian]{babel}
\usepackage[left=2.5cm, right=1.5cm, vmargin=2.5cm]{geometry}
\setlength\parindent{0pt} % Удалить отступы из параграфов.

\usepackage{listings} %% собственно, это и есть пакет listing.
\usepackage{caption}
\DeclareCaptionFont{white}{\color{white}} % это сделает текст заголовка.
\DeclareCaptionFormat{listing}{\colorbox{gray}{\parbox{\textwidth}{#1#2#3}}}
%\captionsetup[lstlisting]{format=listing,labelfont=white,textfont=white}
%\renewcommand\labelenumi{\theenumi)}



\begin{document}
	\lstset{ %
		language=java,                 % выбор языка для подсветки (здесь это java).
		basicstyle=\small\sffamily, % размер и начертание шрифта для подсветки кода.
		numbers=left,               % где поставить нумерацию строк (слева\справа).
		numberstyle=\tiny,           % размер шрифта для номеров строк.
		stepnumber=1,                   % размер шага между двумя номерами строк.
		firstnumber=1,
		numberfirstline=true
		numbersep=5pt,                % как далеко отстоят номера строк от подсвечиваемого кода.
		backgroundcolor=\color{white}, % цвет фона подсветки - используем \usepackage{color}.
		showspaces=false,            % показывать или нет пробелы специальными отступами.
		showstringspaces=false,      % показывать или нет пробелы в строках.
		showtabs=false,             % показывать или нет табуляцию в строках.
		frame=single,              % рисовать рамку вокруг кода.
		tabsize=2,                 % размер табуляции по умолчанию равен 2 пробелам.
		captionpos=t,              % позиция заголовка вверху [t] или внизу [b].
		breaklines=true,           % автоматически переносить строки (да\нет).
		breakatwhitespace=false, % переносить строки только если есть пробел.
		escapeinside={\%*}{*)}   % если нужно добавить комментарии в коде.
	}

	\begin{titlepage}


		\center % Сделать все по-центру.

		%----------------------------------------------------------------------------------------
		%	Верхний колонтитул.
		%----------------------------------------------------------------------------------------

		ФЕДЕРАЛЬНОЕ ГОСУДАРСТВЕННОЕ АВТОНОМНОЕ ОБРАЗОВАТЕЛЬНОЕ УЧРЕЖДЕНИЕ ВЫСШЕГО ОБРАЗОВАНИЯ\linebreak
		«Санкт-Петербургский политехнический университет Петра Великого»\\[2cm] % Название университета.
		\textsc{\Large Институт компьютерных наук и технологий}\\[6.5cm] % Название кафедры.

		%----------------------------------------------------------------------------------------
		%	Заголовок.
		%----------------------------------------------------------------------------------------

		{ \huge \bfseries Курсовая работа	\\
			\Large \mdseries “Реализация Системы Массового Обслуживания”}\\[6.5cm] % Заголовок документа.

		%----------------------------------------------------------------------------------------
		%	Автор.
		%----------------------------------------------------------------------------------------
		\begin{multicols}{2}
			\begin{flushright} \large

				{Выполнил студент группы: 3530904/00104:}\\[0.5cm]

				{Преподаватель:\\}

			\end{flushright}
			\begin{flushright}

				{Почернин В. С.}\\[0.5cm] % Мое имя.


				Смирнов Н. Г. % Имя преподавателя.

			\end{flushright}
		\end{multicols}

		%----------------------------------------------------------------------------------------
		%	Дата.
		%----------------------------------------------------------------------------------------
		\flushright{
			{\today}\\[0.5cm]
		}
		\centering{
			Санкт-Петербург\\
			2022
		}

		\vfill % Заполнить остаток страницы пустым местом.

	\end{titlepage}

\end{document}